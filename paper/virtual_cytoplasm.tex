\documentclass[journal=jacsat,manuscript=article]{achemso}
% \linespread{1.5}
% \usepackage{amssymb,amsmath,graphics,epsfig}
\usepackage{amsmath}
\usepackage[font=small,labelfont=bf]{caption}
\setkeys{acs}{super = true}
\setcitestyle{super,open={},close={}}
\def\citenumfont{}

% \usepackage[version=3]{mhchem} 
\usepackage{color}
\newcommand{\eqnref}[1]{Eqn.~\plainref{#1}}
\newcommand{\figref}[1]{Fig.~\plainref{#1}}
\newcommand{\tabref}[1]{Table~\plainref{#1}}

\title{Cytoplasm Simulation}

\author{David van der Spoel$^1$}

\begin{document}

\maketitle

% \begin{affiliations}
%  \item Science for Life Laboratory, Department of Cell and Molecular Biology. Uppsala University, SE-751 05 Uppsala, Sweden
%  \item Department of Biological Science and Engineering, School of Chemistry and Biological Engineering, University of Science and Technology Beijing, 100083 Beijing, China
% \end{affiliations}

\begin{abstract}
%For Nature, the abstract is really an introductory paragraph set
%in bold type.  This paragraph must be ``fully referenced'' and
%less than 180 words for Letters.  This is the thing that is
%supposed to be aimed at people from other disciplines and is
%arguably the most important part to getting your paper past the
%editors.  End this paragraph with a sentence like ``Here we
%show...'' or something similar.

 
\end{abstract}
\section*{Introduction}

Biomolecules move and function in an environment densely packed with high concentrations of macromolecules. The presence of macromolecules leads to steric effect due to excluded volume effect and intermolecular attrative/repulsive forces due to distributed charges on the surface of macromolecules.

Structure and dynamics of biomolecules are well characterized {\em in vitro}, however how these features differ {\em in vivo} remains unclear. 
Investigating biomolecular properties {\em in vivo} is possible through development in the fields of nuclear magnetic resonance~\cite{reckel2007,pielak2008}, or fluoroscence spectroscopies~\cite{ignatova2004,xie2008,English2011}.
An alternative is to use computational models and simulations techniques. Biomolecular simulations are often carried out under dilute conditions or simple models of macromolecular crowdings~\cite{Spiga2014a,Harada2012a,Nawrocki2017a}. However, more attemps in modeling bacterial cytoplasm have been made recently~\cite{Mcguffee2010,Yu2016a}. 

Here we report on a model of {\em Escherichia coli} cytoplasm at atomistic level. The challenges in modeling and simulation with Molecular Dynamics are pointed and discussed and solutions are provided. 

This work is the first model of {\em E. coli} at atomistic resolution that spans cellular dynamics on a microsecond scale.




\section*{Material and Method}
\subsection*{Cytoplasm model}
In order to investigate the effects of crowding on cellular components in a more realistic way, we built a  cytoplasm model based on {\em Escheria coli} as a model organism~\cite{Dong1996,Bennett2009,Link1997,Mcguffee2010} 

Because of restriction of computational resources, incorporating a full list of proteins, nucleic acids and metabolites in full atomistic model is prohibitive.

In order to build the RNA fraction of our cytoplasm model, we consider data showing that 2.9\% of the dry weight of \textit{E. coli} is composed by tRNAs, which correspond to 74\% of the dry weight of non-ribosomal RNAs, while 55.0\% of it is composed of proteins~\cite{schaechter2006}. Thus, the tRNA weight should correspond to 5\% of the total weight of all protein chains used in our model. Specifically, tRNA(Phe) was selected as a representative due to the availability of a recent crystallographic structure~\cite{Byrne2015}. Copies were added to the simulation box to account for the correct protein/RNA weight ratio (Table X).


The metabolite fraction of our cytoplasm model was build based on data showing that the number of metabolite molecules in the cytoplasm of \textit{E. coli} is about 42.86 times higher than the number of proteins. We considered the most abundant molecule as representative of each metabolite class, i.e. Glutamate for amino acids, ATP for nucleotides, FBP for central carbon intermediates and Glutathione redox cofactors. The copy number for each molecule was calculated as the ratio of their experimentally observed concentration in {\em E. coli}.

\subsection*{Components Preparation}
The proteins and tRNA were downloaded from Protein Data Bank (PDB). We looked for the proteins structures that were either from or expressed in {\em E-coli}. In case of 1U22 (MetE) and 2EIP (Ppa) we used a loop-closure modelng tool based on Random Coordinate Descent (RCD) method~\cite{Chys2013} to correct the information for missing residues. The four metabolites were parametrized using GAFF and Antechamber. 

\subsection*{All-Atom Molecular Dynamics Simulation}

% General Simulation setup:
% \begin{itemize}
% \item Force Field
% \item Water model
% \item Gromacs
% \item Temperature and pressure
% \item (details of soup box)
% \item Minimization 
% \item Equilibration
% \item Production
% \end{itemize}

% \begin{itemize}
% \item Box building
% \item Droplets
% \item Shrinking
% \end{itemize}

 
{\bf General Simulation Setup: }The proteins were simulated at 30\% biomolecular mass fraction in  a physiological salt concentration (0.15M NaCl). For all simulations, Amber99SB-ws force field was used~\cite{Best2014a} in combination with the TIP4P/2005 water model~\cite{Abascal2005b}. Electrostatic interactions were treated using the particle mesh Ewald algorithm~\cite{Essmann1995a}. All chemical bonds were constrained at their equilibrium length using the LINCS algorithm~\cite{Hess2008b} allowing an integration time step of 2 fs. Temperature was controlled at 310 K using the v-rescale algorithm~\cite{Bussi2007a} and a coupling time of 0.5 ps. The pressure was controlled at 1 bar using the Parrinello-Rahman algorithm~\cite{Parrinello1981a} with a time constant of 10 ps.  

\begin{table}[h!]
\label{tbl:component}
\centering
\begin{tabular}{lcc}
\hline
Class & Name (PDB ID) & Number\\
\hline
8*Protein & TufA (1DG1~\cite{Abel1996}) & 6\\
  & MetE (1U22~\cite{Ferrer2004}) & 7\\
  & IcdA (1P8F~\cite{Mesecar2000}) & 2\\
  & AhpC (1YEP~\cite{Parsonage2005}) & 1\\
  & CspC (1MJC~\cite{Schindelin1994}) & 3\\
  & Ppa (2EIP~\cite{Kankare1996}) & 1\\
  & GapA (1S7C~\cite{ShinXXX}) & 1\\
  & Eno (1E9I~\cite{Kuhnel2001}) & 1\\
\hline
RNA & tRNA$^{\text{Phe}}$ (4YCO~\cite{Byrne2015}) & 5\\
\hline
4*Metabolite & GLU & 1436\\
  & ATP & 144\\
  & FBP & 225\\
  & GSH & 255\\
\hline
3*Inorganic Ion & K$^{+}$ & 4602\\
  & Mg$^{2+}$ & 400\\
  & Cl$^{-}$ & 1320\\
\hline
Solvent & Water & 306221\\
\hline
\end{tabular}
\caption{Cytoplasm Components}
\end{table}

Initially, each component was prepared with the same simulation setup as cytoplasm in a box of water. After minimization and equilibration steps, each component together with a layer of water molecule with specific size were extracted. The size of the water shell was probed to make up the total number of water molecules the model needed to make a total biomolecular concentration of 30\%. Possible ions in the water shell were not discarded and at the end counted for the total ionic strength of the cytoplasm model. The components were placed in a preliminary box of size $L=30 nm$ according to their abundancy. The resulting box were equilibrated with temperature and pressure coupling for total time of $200 ns$, this step is called shrinking in this text. The size of the box shrank to $L=22.90 nm$ after $150 ns$.

For error analysis, each simulations were repeated three times with independent starting velocities.

All simulations were performed with Gromacs 2018. Single simulations were started from crystal conformations. The cytoplasm simulations starting conformation were taken from the equilibrated conformation of each single simulation.

{\bf Single Simulation Setup: } Each component was simulated with the same parameter as the cytoplasm for $200 ns$.

\subsection*{Analysis}
Before any analysis the periodic boundary condition (pbc) artifacts have been removed. We used GROMACS tools to do the analysis. For single component simulations, first the components were made whole and jump removed and then all the atom were put inside the compact box. The same treatment were applied to the cytoplasm simulations. Additionally, each component's trajectory were extracted and fitted by rotation and translation for later rotational correlation time analysis. 


A Mean Square Displacement (MSD) analysis was used to calculate the translational diffusion coefficient~\cite{Allen1987a}. The diffusion coefficients were extracted by a linear fit to MSD analysis by averaging blocks with a length of $10 \,ns$. In principle diffusion coefficient needs to be corrected for finite size effects~\cite{Yeh2004} but due to relatively large simulation boxes this correction is negligible.

% We followed the Lipari-Szabo approach to calculate the rotational correlation time, using a second order Legendre polynomial $P_2({\mathbf r}_{NH})$ to be able to compare the results to experimental relaxation data~\cite{Lipari1982b}. First the average order parameter $<\!\!S^2\!\!>$ over all ${\mathbf r}_{NH}$ vectors for each protein was determined from trajectories after the biomolecular rotation- and translation-movements were removed. Then the rotational correlation function $C(t)$ of the original trajectories was fitted to 
% \begin{equation}
% C(t) ~=~ <\!\!S^2\!\!> \exp(-t/\tau_M) + (1-<\!\!S^2\!\!>)\exp(-t/\tau_T)
% \end{equation}
% where $\tau_M$ is the rotational correlation time (tumbling) and $\tau_T = (\tau_M^{-1} + \tau_e^{-1})^{-1}$, where $\tau_e$ is the internal bond movement. 




Protein protein interaction: Who is interacting with who?! (Residue)



\section*{Results}

\begin{itemize}
	\item Translational Diffusion
		\begin{itemize}
			\item Translation diffusion as a function of size for both soup components and ingredients in the same plot
			\item D/D0 for all components
			% \item D-x, D-y, D-z for metabolites, then a combination of 2-d Diffusion
		\end{itemize}
	% \item Rotational Diffusion
	% 	\begin{itemize}
	% 		\item Rotational correlation time for non-metabolite components and comparison with ingredients
	% 		\item C-R/C-R0
	% 	\end{itemize}
	\item RMSD analysis
		\begin{itemize}
			\item Comparison of RMSD in  the soup and ingredients
			\item RMSD for metabolites around macromolecules
		\end{itemize}
	\item Protein-Protein interaction
	\item Metabolite-Protein interaction
\end{itemize}

 
\section*{Discussion}\label{sec:dissc}
 

\bibliography{../bibtex/library}


 

% \begin{addendum}
%  \item  
%  \item[Competing Interests] 
%  \item[Correspondence]  
% \end{addendum}
 
\end{document}


% Cytoplasm setup:
% \begin{itemize}
% \item Constitutents of the cytoplasm in ratio, (non-ribomosal) protein, (non-ribosomal) RNA and metabolites {\bf review paper}
% \item Proteins fraction 
% \item Nucleic Acid fraction
% \item Metabolites fraction
% \item Physiological condition of the cell
% \item Biomolecular fraction 
% \end{itemize}

% General Simulation setup:
% \begin{itemize}
% \item Force Field
% \item Water model
% \item Gromacs
% \item Temperature and pressure
% \item (details of soup box)
% \item Minimization 
% \item Equilibration
% \item Production
% \end{itemize}

% \begin{itemize}
% \item Box building
% \item Droplets
% \item Shrinking
% \end{itemize}



% {\it Cytoplasm Model: }The most abundant non-ribosomal proteins that account for 50\% of the {\em Escherichia coli} xxx cytoplasm's protein composition were chosen for our cytoplasm model~\cite{}. Specifically TufA (pdb 1dg1), MetE (pdb 1u22), IcdA (pdb 1p8f), AhpC(pdb 1yep), CspC (pdb 1mjc), Ppa (pdb 2eip), GapA (pdb 1s7c) and .


 
% The proteins were simulated at 30\% biomolecular mass fraction in  a physiological salt concentration (0.15M NaCl).


% For all simulations the Amber99sb-ildn force field was used~\cite{Hornak2006, Lindorff2010} in combination with the TIP4P/2005 water model~\cite{Abascal2005b}. Electrostatic interactions were treated using the particle mesh Ewald algorithm~\cite{Essmann1995a} and XXXX. All chemical bonds were constrained at their equilibrium length using the LINCS algorithm~\cite{Hess2008b} allowing an integration time step of 2 fs. Temperature was controlled at 310 K using the v-rescale algorithm~\cite{Bussi2007a} and a coupling time of 0.5 ps. The pressure was controlled at 1 bar using the Parrinello-Rahman algorithm~\cite{Parrinello1981a} with a time constant of 10 ps.  

% Among proteins, (the corrections)
% The metabolites structures have been prepared by ... .




% For tRNA and metabolites the vector ... was used.



% \begin{table}
% \centering
% \caption{Proteins PDB ID}
% \label{tbl:pdb}
% \begin{tabular}{lcc}
% \hline
% Protein (pdb)  \\
% \hline
% crowder\_n1 & 1dg1~\cite{Abel1996}   \\ 
% crowder\_n2 & 1u22~\cite{Ferrer2004} \\
% crowder\_n3 & 1p8f~\cite{Mesecar2000} \\
% crowder\_n4 & 1yep~\cite{Parsonage2005}\\
% crowder\_n5 & 1mjc~\cite{Schindelin1994}\\
% crowder\_n6 & 2eip~\cite{Kankare1996}\\
% crowder\_n7 & 1s7c~\cite{ShinXXX}\\
% crowder\_n8 & 1sl4~\cite{Guo2004d}\\
% crowder\_n9 & 1wz2~\cite{FukunagaXXX}\\
% \hline
% \end{tabular}
% \end{table}




% Cytoplasm setup:
% \begin{itemize}
% \item Constitutents of the cytoplasm in ratio, (non-ribomosal) protein, (non-ribosomal) RNA and metabolites {\bf review paper}
% \item Proteins fraction 
% \item Nucleic Acid fraction
% \item Metabolites fraction
% \item Physiological condition of the cell
% \item Biomolecular fraction 
% \end{itemize}
 